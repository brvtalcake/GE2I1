\documentclass[french]{article}

\usepackage{graphicx}
\usepackage{listings}
\usepackage{xcolor}
\usepackage{hyperref}
\usepackage[skip=10pt, indent=0pt]{parskip}

\usepackage[T1]{fontenc}    % Encodage T1 (adapté au français)
\usepackage{lmodern}        % Caractères plus lisibles
\usepackage{babel}

\newcommand{\ifnotext}[3]{%
  \sbox0{#1}%
  \ifdim\wd0=0pt
    {#2}% if #1 is empty
  \else
    {#3}% if #1 is not empty
  \fi
}

% Make it bold and a bit smaller
\newcommand{\fixmetodomaybetext}[1]{\ifnotext{#1}{}{: #1}}
\newcommand{\FIXME}[1]{\small\textbf{FIXME\fixmetodomaybetext{#1}}\normalsize}
\newcommand{\TODO}[1]{\small\textbf{TODO\fixmetodomaybetext{#1}}\normalsize}

\newcommand{\feuille}[1]{\chapter{#1}}
\newcommand{\theme}[1]{\section{#1}}
\newcommand{\exercice}[2]{\subsection[Exercice \no#1]{Exercice \no#1 - #2}}
\newcommand{\question}[1]{\subsubsection{Question \no#1}}

\hypersetup{
    breaklinks,
    colorlinks=false,
    pdfborder={0 0 0}
}

\title{Exercices Corrigés d'OML}
\author{Axel PASCON}
\date{\today}

\begin{document}

\maketitle

\tableofcontents

\feuille{Exercices de grosse feuille de TD}
\TODO{Nom de chapitre davantage approprié}

\theme{Fonctions de base}

\exercice{1}{Droites}

\question{1}
\TODO{}
\question{2}
\TODO{}

\exercice{4}{Courbes à connaître}
\question{1}

(1) --> \( y = 1 \)
(2) --> \( y = x^2 \)
(3) --> \( y = 1 \)

\TODO{Finish it with a proper table instead}

\exercice{12}{Vidange d'un réservoir}

\question{1}

Si \(h\_{0}\) augmente, alors \(t\_{0}\) augmente.
Si \(S\_{1}\) augmente, alors \(t\_{0}\) augmente.
Si \(S\_{2}\) augmente, alors \(t\_{0}\) diminue.

\question{2}
dev-perl/Template-Plugin-Latex
(b), (c) et (d) ne correspondent pas aux observations précédentes.
Par ailleurs, certaines violent les règles d'homogénéité.

Par déduction, seule (a) est possible.

\question{3}

\TODO{Tracer les courbes (apprendre à le faire, déjà\dots lol)}

\exercice{1}{Pulsation et Amplitude d'un signal}

\question{1}

Fonction définie par \(f = 2\sin(t)\): la plus "grande" / ample des courbes
Fonction définie par \(g = \sin(2t)\): la dernière, par déduction
Fonction définie par \(h = \sin(4t)\): la plus "rapide" des courbes, celle avec la plus petite période

\exercice{2}{Pulsation et Amplitude d'un signal}

\question{1}

Fonction définie par \(m(t) = 2\cos(0.5t - 1.5)\): \TODO{}
Fonction définie \dots 

\end{document}